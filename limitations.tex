\documentclass[./discussion]{subfiles}

\begin{document}

-Les cas étudiés étant rétrospectifs, le manque de précision dans la collecte de données pourrait affecter leur analyse. En effet, des paramètres mal rapportés peuvent modifier la description de la pathologie. Une dyspnée, une hématémèse, une douleur non précisée dans une nécrose œsophagienne peut changer la fréquence et la pondération du paramètre considéré.
-Dans la nécrose œsophagienne compressive, l’absence de séries larges ne permet pas de comparer les traitements afin de les codifier suivant le stade de la pathologie.
-Néanmoins, les résultats des études n=1 patient aident à la prise en charge des cas exceptionnels, s’éloignant des recommandations habituelles. En plus d’éclairer une situation rare, ils ont le potentiel de changer la pratique clinique [28], [29].

\end{document}