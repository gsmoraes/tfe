\documentclass[./tfe.tex]{subfiles}

\begin{document}

La NOA reste une entité pathologique très rare où une partie ou la totalité de l’œsophage développe une nécrose[1], [2].

La clinique est marquée essentiellement par une hématémèse 66\%, du méléna 33\%, une douleur sternale ou abdominale 28\% ainsi qu’une dysphagie 12\% [2]. La fièvre d'origine indéterminée et le choc sont également décrits.

La NOA peut survenir dans divers états pathologiques et dans une population cible des patients masculins de plus de 50 ans en moyenne, porteurs de diabète, d’hypertension artérielle, malnutris, athéromateux, en abus d’alcool ou atteints d’affections néoplasiques [1], [2].

Rarement, la rupture ou la dissection aortique thoracique et leurs traitements en sont la cause [1], [3]–[19]  checar os casos.
Dans ce contexte, la NOA a ses propres caractéristiques. Sur 16 cas répertoriés (Tableau 1), 44\% avaient une dysphagie, 31\% une hématémèse, 25\% de la fièvre, 13\% des nausées et vomissements.

L’antécédent de l’anévrisme aortique thoracique descendant, la récente douleur thoracique dorsale, l’hypotension et la tachycardie réactionnelle ont orienté vers une dissection ou rupture probable de l’anévrisme. Seules l’hématémèse et la dysphagie auraient pu faire le lien et amener à considérer une lésion œsophagienne concomitante chez notre patient âgé, de sexe masculin et atteint d’athérosclérose. La dysphagie seule n’est pas très spécifique. Elle est fréquente chez des sujets âgés de plus de 50 ans ; 16 à 22\% [20]. Par contre son apparition récente dans un contexte de syndrome thoracique aortique est hautement suspecte d’une atteinte œsophagienne.

La fièvre nous semble être un signe évolutif tardif de toute NOA. Dans notre cas ainsi que chez Park [9], elle est apparue après la dysphagie.

La biologie peut être marquée par un syndrome inflammatoire [1], [21], [22]. La majoration de celui-ci associée à un tableau clinique qui se dégrade, devraient faire penser à une complication infectieuse. Habituellement, c’est une infection de prothèse aortique qui pourrait être suspectée. Mais le diagnostic différentiel devrait inclure une NOA.

La radiographie du thorax peut servir de premier élément iconographique permettant d’explorer le médiastin et les poumons. Dans notre situation, elle a montré deux anomalies interpellantes : un épanchement pleural bilatéral et la déviation œsophagienne droite. Cette dernière est suggestive d’une compression œsophagienne extrinsèque au départ du médiastin postérieur[7], [9]. Elle pourrait expliquer la dysphagie. Le patient de Park [9] âgé de 47 ans avec des fractures costales et claviculaire suite à un accident de la voie publique avait présenté une dysphagie soudaine au 6e jour de son hospitalisation. Suite à l’apparition d’un choc hypovolémique, un CT Thoracique avait révélé une rupture traumatique de l’aorte et une déviation œsophagienne droite. Une réparation aortique a été réalisée. Mais, la dysphagie avait persisté et la déviation œsophagienne n’avait pas été prise en compte jusqu’à ce qu’une médiastinite sur perforation œsophagienne ait été tardivement diagnostiquée.

Le diagnostic de certitude est posé par l'OGD[1], [2], qui peut montrer un œsophage avec une muqueuse friable ou noire (Black Esophagus), ou encore une œsophagite érosive, des exsudats jaunâtres, une nécrose, un saignement et une fistulisation éventuelle comme chez notre patient.

Le CT Scanner Thoracique avec prise orale de la gastrographine permet de confirmer la fistule œsophagienne et l’étendue de la médiastinite[21]. Dans notre cas, cet examen a été utilisé à deux reprises. Réalisé à l’admission, sans la gastrographine, il a clairement établi le diagnostic de la dissection aortique type B, de sa rupture, et a décrit l’épanchement pleural et la compression œsophagienne par l’hématome. Ce qui a conduit au placement de TEVAR. Par la suite, devant la mise en avant du tableau de nécrose œsophagienne, il a confirmé à l’aide de la gastrographine, la présence d’une fistule œsophagienne.

Pour comprendre la physiopathologie de la NOA, nous rappelons l’anatomie médiastinale. Le compartiment médiastinal postérieur est composé de [23] : l’aorte thoracique, le canal thoracique, les vaisseaux lymphatiques, les ganglions, la veine azygos et hémi-azygos, l’œsophage et les plexus nerveux. Lorsqu’un organe y est lésé, le volume additionnel qui en résulte peut conduire à l’augmentation de la pression locale avec compression des organes.

Lorsque l’œsophage est lésé par ce mécanisme, nous avons introduit la notion de la nécrose œsophagienne aigüe compressive (NOAC). Si tel n’est pas le cas, il s’agirait de la NOA non compressive (NOAN).

Pour expliquer la pathogenèse de la NOAN, plusieurs auteurs dont Gurvits et col.[1] ont proposé diverses théories. Il existerait trois composantes principales : l’insuffisance hémodynamique, le reflux du contenu gastrique vers l’œsophage et la réduction des capacités de défense de la muqueuse œsophagienne. Habituellement, une paroi œsophagienne saine attaquée par le reflux du contenu chimique gastrique peut se défendre et se régénérer grâce à un flux sanguin normal. Mais en cas d’insuffisance vasculaire, il y aurait inhibition de la neutralisation des radicaux libres intracellulaires entravant la réparation de la muqueuse, ce qui conduit à la mort cellulaire. Cette ischémie peut résulter d’une complication de l’athérosclérose ou d’une vasoconstriction excessive telle celle liée à la cocaïne [2]. Dans la malnutrition et le cancer, les capacités intrinsèques de réparation de la muqueuse œsophagienne seraient touchées. Ce serait le même phénomène qui expliquerait la NOA de l’infection œsophagienne à Herpès Simplex , Cytomégalovirus, Candida Albicans , actinomycose et Klebsielle Pneumoniae [1], [2].
Dans la NOAC, la théorie de Gurvits ne suffit plus. Il faudrait la compléter par d’autres hypothèses. Elles sont basées surtout sur la pathologie de l’aorte thoracique et le volume additionnel crée par sa rupture traumatique[9], l’anévrisme évolutif[4], [8], sa dissection [6], [14]–[17] sa rupture [3]–[5], [7], [10]–[13] et aussi son traitement par le TEVAR [3]–[8], [10]–[14].

Hypothèses physiopathologiques de la NOAC :
1) le syndrome du compartiment médiastinal [5], [9]–[11], [15], [17], [19], [22], [24] ;
2) l’occlusion des branches artérielles aortiques oesophagiennes causée par la thrombose de l'hématome aortique transmural [6] ou par le TEVAR [8]–[10], [19] [25] checar;
3) l’ischémie causée par une dissection qui sépare le flux dans un système haut et bas flux sanguin [7], [8].

Dans la littérature, les phénomènes les plus souvent évoqués sont le syndrome de compartiment médiastinal et l’occlusion des branches artérielles aortiques.

Dans le syndrome du compartiment médiastinal, la masse volumique de l'hématome médiastinal postérieur conséquent à la rupture aortique jouerait un rôle majeur. Elle augmenterait la pression médiastinale entraînant une compression des organes avoisinants. Le réseau artériolaire, capillaire et veineux œsophagien serait incapable de surmonter ces tensions intra médiastinales. Il s’en suivrait une réduction de l’apport de l’oxygène responsable de la mort cellulaire[22] . Ce mécanisme pourrait s’appliquer à notre patient. Et selon la théorie de Gurvits sur l’attaque de la muqueuse œsophagienne par le reflux, notre patient avait aussi une hernie hiatale qui aurait pu accentuer ce phénomène.
De surcroit, le patient est resté 24h sous traitement médical avant la mise en place du TEVAR. Cela a augmenté le temps d’ischémie œsophagienne à laquelle sont venues s’ajouter d’hypothétiques lésions liées au TEVAR.
A part la compression pariétale œsophagienne directe, l’occlusion d’artères nourricières œsophagiennes de calibre plus important peut se produire. L’hématome pariétal de la fausse lumière anévrismale peut obstruer l’émergence des artères œsophagiennes[6]–[8]. Cette hypothèse occlusive artérielle est également corroborée par Joubert et al [7] qui a décrit le cas d'un patient ayant présenté une NOA trois jours après une dissection aortique Stanford B, traité médicalement. L'hématome thrombosé aurait obstrué l’émergence des branches aortiques œsophagiennes provoquant l'accumulation de radicaux libres entrainant ainsi la mort cellulaire.

Dans notre cas, nous pensons que les lésions d’œsophagite ulcéreuse extensive et fistulisée qui s'étendait dès 22 cm au cardia, c'est-à-dire toute l'extension œsophagienne vascularisée par l'aorte et ses branches seraient compatibles avec cette hypothèse.

La pose du TEVAR en soi peut aussi entrainer une obstruction des artères nourricières [8], [10]. Mais, reste cela difficile à distinguer d’une obstruction par l’hématome pariétal vu que le traitement endovasculaire laisse celui-ci en place.

Sur le plan thérapeutique, pour la NOAN, la prise en charge est essentiellement médicale. Il faudrait tenir compte du degré d’atteinte œsophagienne stadifiée par Gurvits, la présence d’une éventuelle médiastinite ainsi que l’état général du patient. Le traitement est assuré par [1], [2] : la réhydratation, les inhibiteurs de pompe à proton, le sucrasulfate, la transfusion, l’antibiothérapie et la nutrition parentérale. La mise en place d’une sonde nasogastrique est déconseillée vu les risques d’aggravation de lésions œsophagiennes [1].

L’évolution est favorable dans plus de 60\% des cas. En cas d’échec ou de médiastinite, il faudrait recourir à la chirurgie.
Pour la prise en charge thérapeutique de la NOAC, il faudrait considérer les différents problèmes cliniques. En effet, le mécanisme lésionnel séquentiel est d’abord une atteinte aortique traumatique ou une dissection et rupture anévrismale aortique thoracique suivie d’une nécrose et fistule œsophagienne.
La prise en charge de la dissection aortique type B de Stanford est codifiée. Le TEVAR est réservé à la dissection compliquée de rupture, en cas de mauvaise perfusion ou d’absence de réponse au traitement médical[25]. L’anévrisme de notre patient s’était rompu et avait entraîné une mauvaise perfusion rénale avec un débit de filtration glomérulaire de 26 mL/min/1.73 \si{\cubic\meter} ( normale > 90 mL/min/1.73 \si{\cubic\meter}), ce qui a justifié le placement du TEVAR. Il est actuellement bien établi que l’approche endovasculaire a une morbimortalité inférieure à la chirurgie conventionnelle. A titre d’exemple, l’on observe une réduction de la mortalité postopératoire intra-hospitalière de 33.9\% à 10\% [25]. 
Une fois la pathologie aortique contrôlée, l’atteinte œsophagienne souvent méconnue et de diagnostic tardif doit être prise en charge chirurgicalement [3]–[8], [10]–[14]. En plus de l’antibiothérapie, les gestes à réaliser sont le débridement médiastinal et pleural, le drainage de l’hématome, l’exclusion œsophagienne ou l’œsophagectomie.

Dans notre cas, vu la cachexie et la dégradation de l’état général, nous avons opté pour une exclusion œsophagienne après la mise en place du TEVAR.

Décrite par Johnson en 1956 [26], l’exclusion œsophagienne est une procédure rarement utilisée. Elle est surtout indiquée lorsqu’une fistule œsophagienne persiste avec une médiastinite extensive, chez des patients récusés pour une thoracotomie [26]. En présence d’adhérences médiastinales peropératoires rédhibitoires [12], [22], elle pourrait être proposée.

La technique de l’exclusion œsophagienne a été déjà précisée [26]. Une incision cervicale gauche oblique est pratiquée le long du bord supérieur du muscle sternocléidomastoïdien. Le muscle omohydien est identifié et sectionné. Mise en évidence de la veine jugulaire interne. La dissection se poursuit vers le bord médian de la gaine carotidienne. Si nécessaire, la veine thyroïdienne moyenne et l’artère thyroïdienne inférieure peuvent être liées. Une courte sonde naso-œsophagienne provisoire permet de palper l’œsophage entre la trachée en avant et la colonne cervicale postérieurement. Une mobilisation minutieuse circulaire de l’œsophage permet d’éviter la lésion du nerf récurrent. Passage d’un lacs autour de l’œsophage, section œsophagienne en gardant une longueur cervicale proximale suffisante pour l’œsophagostomie. Par la suite, réalisation d’une laparotomie médiane, agrafage de la jonction œsogastrique, pyloroplastie, gastrostomie de décompression et confection d’une jéjunostomie d’alimentation.

L’objectif est d’arrêter de souiller le médiastin par de la salive riche en germes et par le contenu gastrique via l’œsophage perforé. La jéjunostomie sert à l’alimentation et la gastrostomie ainsi que la pyloroplastie assurent la vidange gastrique. La remise en continuité peut se faire à distance avec un côlon ou l’estomac.

Chez Rohatgi [26], l’exclusion œsophagienne avait été pratiquée chez 6 patients, et le taux de succès était de 50\%. Ce résultat était intéressant au vu de près de 100\% de mortalité qu’il estimait liée au sepsis de la médiastinite.

Parfois, lors de ce traitement, le médiastin n’est pas débridé et l’hématome est laissé en place. Ceci pourrait entretenir l’hyperpression médiastinale et la médiastinite. Et selon Paramesh [27] existerait un risque de reperméabilisation oesogastrique spontané.

Par conséquent, la meilleure approche chirurgicale semble être l’œsophagectomie. Elle a assuré la survie chez 4 des 6 patients opérés (66.66\%), par cette technique dans un contexte de NOAC. Les 2 patients décédés avaient eu, par le même auteur, une chirurgie avec ablation de TEVAR. A notre avis, il ne semble pas opportun de procéder ainsi à une chirurgie trop longue et extensive. Le geste supplémentaire d’ablation de TEVAR pourrait être la cause du décès.
D’autres traitements tels, la pose de drain de Kehr et la prothèse œsophagienne se sont soldées par des décès dans cette revue des cas de NOAC (voir tableau 2). La mortalité était de 100\% en l’absence de traitement chirurgical.

Le traitement des perforations œsophagiennes après TEVAR avec l'utilisation du stent œsophagien est un traitement palliatif, car ces patients doivent être traités par antibiothérapie pour le reste de leur vie afin de prévenir la progression de l'infection de l'endoprothèse, ces patients ont un taux de survie de 17\% à un an [22].

Mise à part la technique chirurgicale, la survie peut dépendre aussi de la précocité du diagnostic permettant une prise en charge avant l’installation de la nécrose transmurale et la perforation.

\subsection*{Limitations de l’étude}
\subfile{./limitations.tex}

\subsection*{Conclusion}
\subfile{conclusion.tex}

\end{document}