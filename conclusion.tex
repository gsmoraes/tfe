\documentclass[./discussion.tex]{subfiles}

\begin{document}

Malgré sa rareté, la nécrose œsophagienne aiguë devrait faire partie du diagnostic différentiel devant toute hématémèse, méléna et dysphagie associée aux facteurs d’insuffisance circulatoire. Souvent, le traitement est médical avec un pronostic favorable dans plus de 60% des cas.

Dans la NOA de la pathologie aortique thoracique et de son traitement, le diagnostic est souvent tardif au stade de fistule et médiastinite. La prise en charge est dès lors chirurgicale. Et, la technique appropriée, si l’état général du patient l’autorise, semble être une œsophagectomie par thoracotomie afin d’enlever la source de l’infection et de nettoyer le médiastin.

Pour améliorer le pronostic, nous préconisons une endoscopie digestive haute de principe à réaliser précocement et systématiquement. Cette promptitude de diagnostic associée à un traitement précoce permettrait d’améliorer la survie. 

\end{document}