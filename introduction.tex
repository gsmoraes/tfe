\documentclass[./tfe.tex]{subfiles}

\begin{document}

La nécrose œsophagienne aigue (NOA) est une affection rare avec une incidence de 0,01\% à 0,28\% |1|. Sa première description a été faite par Goldenberg en 1990|1|. Elle touche essentiellement des sujets masculins d’âge moyen de 70 ans, porteurs de comorbidités. Survenant dans une situation de bas débit sanguin, elle se manifeste principalement sous forme d'hématémèse et de méléna, mais aussi de dysphagie ou de fièvre d'origine indéterminée. Dans 5\% des cas, la présentation clinique est marquée par le décès dès l’admission|2|. Son diagnostic est établi par une œsophago-gastro-duodénoscopie (OGD) qui démontre une coloration noire de la muqueuse œsophagienne (Black Esophagus) surtout distale (51\%) ; épargnant souvent la jonction œso-gastrique |2,9| checar os casos. Son traitement est essentiellement supportif : le remplissage vasculaire, la transfusion, l’administration de sucrasulfate et des inhibiteurs de pompe à protons.

L’évolution de la NOA est marquée souvent par une résolution complète dans 60\% des cas. Environ 5\% des patients peuvent présenter une perforation œsophagienne|2|.

La mortalité globale de NOA est de 32\%|2|. Et, en cas de médiastinite sévère, elle peut atteindre 40\% |1|. Les facteurs péjoratifs associés à la mortalité sont : la perforation œsophagienne, une septicémie, un choc hypovolémique, cardiogénique, les facteurs de risque d’athérosclérose ou encore la présence d’une tumeur maligne.

Dans ce travail, nous présentons un cas de NOA apparu après une dissection aortique type Stanford B ayant évolué vers une perforation. Notre objectif est de discuter de la clinique, des mécanismes physiopathologiques responsables de la nécrose, de ses complications, et de la place de la chirurgie tant aortique qu’œsophagienne.

\end{document}